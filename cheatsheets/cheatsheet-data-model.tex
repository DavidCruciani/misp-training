\begin{center}{
    \huge{\textbf{MISP Data Model Cheat Sheet}}}\\
\end{center}

\begin{multicols*}{3}
    \begin{minipage}{0.3\textwidth}
        \begin{itemize}[noitemsep,topsep=2pt,parsep=0pt,partopsep=0pt]
            \item[\taggable] Context such as \taxonomies or \clusters can be attached to the element
            \item[\distributable] Has a distribution level 
            \item[\synchronisable] Can be synchronised to/from other instances
        \end{itemize}
    \end{minipage}
    \vspace*{0.5em}

    % EVENT 
    \cheatbox[\faicon{envelope}]
        [Group datapoints and context together. Acting as an envelop, it allows setting distribution and sharing rules for itself and its children.]
        [Encode incidents/events/reports/…]
        [\taggable \distributable \synchronisable]
        [Encapsulations for contextually linked information.]
        {\linkdest{event}Event}
        {
            $\blacktriangleright$ \events can contain other elements such as \attributes, \objects and \eventreports.\\
            $\blacktriangleright$ The distribution level and any context added on an \event (such as \taxonomies) are propagated to its underlying data.
        }

    % ATTRIBUTE 
    \cheatbox[\faicon{cube}]
        [Individual data point. Can be an indicator or supporting data.]
        [Domain, IP, link, sha1, attachment, …]
        [\taggable \distributable \synchronisable]
        [Basic building block to share information.]
        {\linkdest{attribute}Attribute}
        {
            $\blacktriangleright$ \attributes cannot be duplicated inside the same \event and can have \sightings.\\
            $\blacktriangleright$ The difference between an indicator or supporting data is usualy indicated by the state of the attribute's \texttt{to\_ids} flag.
        }

    % Object 
    \cheatbox[\faicon{cubes}]
        [Groups \attributes that are intrinsically linked together.]
        [File, person, credit-card, x509, device, …]
        [\distributable \synchronisable]
        [Advanced building block providing \attribute compositions via templates.]
        {\linkdest{object}MISP Object}
        {
            $\blacktriangleright$ \objects have their attribute compositions described in their respective template. They are instanciated with \attributes and can \reference other \attributes or \objects.\\
            $\blacktriangleright$ MISP is not required to know the template to save and display the object. However, \textit{edits} will not be possible as the template to validate against is unknown.
        }
    \columnbreak

    % Object Reference
    \cheatbox[$\nearrow$]
        [Allows to create relationships between entities, thus creating a graph where they are the edges and entities are the nodes.]
        [Represent behaviours, similarities, affiliation, …]
        [\synchronisable]
        [Relationships between individual building blocks.]
        {\linkdest{reference}Object Reference}
        {
            $\blacktriangleright$ \references can have a textual relationship which can come from MISP or be set freely.
        }

    % Sightings
    \cheatbox[\faicon{eye}]
        [Allows to add temporality to the data.]
        [Record activity or occurence, perform IoC expiration, …]
        [\synchronisable]
        [Means to convey that an \attribute has been seen.]
        {\linkdest{sighting}Sightings}
        {
            $\blacktriangleright$ \sightings are the best way to express that something has been seen. They can also be used to mark \textit{false positives}.
        }

    % Event report
    \cheatbox[\faicon{file-text}]
        [Supporting data point to describe events or processes.]
        [Encode reports, provide more information about the \event, …]
        [\distributable \synchronisable]
        [Advanced building block containing formated text.]
        {\linkdest{eventreport}Event Report}
        {
            $\blacktriangleright$ \eventreports are markdown-aware and include a special syntax to reference data points or context.
        }

    % Proposals
    \cheatbox[\faicon{comment}]
        [Allow the correction or the creation of \attributes for \events your organisation does not own.]
        [Disable the IDS flag, Correct errors]
        [\synchronisable]
        [Clone of an \attribute containing information about modification to be done.]
        {\linkdest{proposal}Proposals}
        {
            $\blacktriangleright$ As \proposals are sync., if the creator organisation is connected to the MISP instance from where the \proposal has been created, it will be able to either \textit{accept} or \textit{discard} it. 
        }
    \columnbreak

    % Taxonomies
    \cheatbox[$\mathcal{T}$]
        [Enable efficent classification globally understood, easing consumption and automation.]
        [Provide classification such as: TLP, Confidence, Source, Workflows, Event type, …]
        []
        [Machine and human-readable labels standardised on a common set of vocabularies.]
        {\linkdest{taxonomy}Taxonomies}
        {
            $\blacktriangleright$ Even though MISP allows the creation of free-text tags, it's always preferable to use those coming from \taxonomies, if they exists.
        }

    % Galaxies
    \cheatbox[\faicon{rebel}]
        [Bundle \clusters by their type to avoid confusion and to ease searches.]
        [Bundle types: Exploit-Kit, Preventive Measures, ATT\&CK, Tools, Threat-actors, …]
        []
        [Act as a container to group together context described in \clusters by their type.]
        {\linkdest{galaxy}Galaxies}
        {}

    % Galaxy Clusters
    \cheatbox[\faicon{rebel}]
        [Enable description of complex high-level information for classification.]
        % [\texttt{threat-actor="APT 29"}, \texttt{country="germany"}, \texttt{mitre-attack-pattern="Disk Wipe - T1561"}]
        [Extensively describe elements such as: threat actors, countries, technique used, …]
        [\distributable \synchronisable]
        [Kownledge base items used as tags with additional complex meta-data aimed for human consumption.]
        {\linkdest{cluster}Galaxies Clusters}
        {
            $\blacktriangleright$ \clusters can be seen as an enhanced \taxonomy as they can have meta-data and relationships with other \clusters.\\
            $\blacktriangleright$ Any \clusters can contain the following:
            \begin{itemize}[noitemsep,topsep=2pt,parsep=0pt,partopsep=0pt]
                \item \texttt{Cluster Elements}: Key-Value pair forming the meta-data.
                \begin{itemize}[noitemsep,topsep=2pt,parsep=0pt,partopsep=0pt]
                    \item[Example:] \texttt{Country:LU}, \texttt{Synonym:APT28}, \texttt{Currency:Dollar}, \texttt{refs:https://*}, …
                \end{itemize}
                \item \texttt{Cluster Relations} (\taggable\synchronisable\distributable): Enable the creation of relationships between one or more \clusters.
                \begin{itemize}[noitemsep,topsep=2pt,parsep=0pt,partopsep=0pt]
                    \item[Example:] Threat actor \texttt{X} \texttt{is similar} to threat actor \texttt{Y} with \texttt{high-likelyhood.}
                \end{itemize}
            \end{itemize}
        }
\end{multicols*}