\begin{center}{
    \huge{\textbf{MISP Concepts Cheat sheet (FR)}}}\\
\end{center}

\begin{multicols*}{2}
\cheatboxlarge{Glossary}{
    \boxentry{Correlations}{Relations créées automatiquement depuis un \attribute. Elles permettent l'inter-connexion entre \events basés sur leurs \attributes.}
    \boxentry{Correlation engine}{Système utilisé par MISP pour créer des correlations entre la valeur des \attribute. Il supporte actuellement la comparaison stricte de chaines de caractères, SSDEEP et les blocks CDIR.}
    \boxentry{Caching}{Processus de récupération de données d'une instance ou d'un feed afin de sauver les hashs des valeurs récupérées servant à la corrélation et la recherche.}    
    \boxentry{Delegation}{Acte de transférer la propriété d'un event vers une autre organisation tout en cachant le créateur original afin de garantir l'anonymat}    
    \boxentry{Deletion (hard/soft)}{\textit{Hard deletion} est l'acte de supprimer un element du système; Cela ne va pas révoquer la donnée sur les autres systèmes contrairement à la \textit{Soft deletion} où la révocation est propagée sur le réseau d'instances connectées.}
    \boxentry{Extended Event}{\event qui en étend un autre, permetant d'avoir une vue combinée. L'organisation qui a étendu l'\event est le propriétaire de l'extension . Cela permet à n'importe qui d'étendre n'importe quel \events et d'en avoir le contrôle.}
    \boxentry{\galaxy Matrix}{Matrice dérivée d'un \clusters appartenant à la même \galaxy. La structure (pages et colones) est définie au niveau de la \galaxy et son contenu provient des méta-données des \clusters.}
    \boxentry{Indicators}{\attribute contenant un pattern utile pour détecter une activité suspicieuse ou malveillante. Ils ont souvent la propriété \texttt{to\_ids} activée.}
    \boxentry{Orgc / Org}{\textit{L'organisation créatrice} (\textbf{Orgc}) est l'organisation qui a créé les données et qui est la seule à pouvoir les modifier. \textit{L'organisation propriétaire} (\textbf{Org}) est l'organisation qui possède les données et qui peut consulter le contenu. \textbf{Orgc} \& \textbf{Org} ne sont pas nécessairement les mêmes organisations.}
    \boxentry{Publishing}{Action de déclarer qu'un \event peut être synchronisé. Ce processus peut aussi envoyer des notifications et permet certains types de format d'export.}
    \boxentry{Pulling}{Action d'utiliser un utilisateur depuis une autre instance pour récupérer les données accessibles et les stoquer localement.}
    \boxentry{Pushing}{Action d'utiliser une connexion via un \textit{sync. user} pour envoyer des données à une autre instance.}
    \boxentry{Synchronisation}{Est l'échange de données entre deux (ou plus) instances MISP par le mécanisme de \textit{pull} ou \textit{push}.}
    \boxentry{Sync. filtering rule}{Peuvent être appliquées sur un lien de synchronisation pour le \textit{pull} ou \textit{push} afin de bloquer ou pemettre à des données d'être transférées.}
    \boxentry{Sync. User}{Rôle spécial pour un utilisateur donnant des permissions de synchronisation supplémentaires. L'utilisation des \textit{sync users} est la manière recommandée de configurer la synchronisation via \textit{push}.}
    \boxentry{Proposals}{Mécanisme pour proposer des modifications à l'organisation créatrice (\textbf{Orgc}). Si un chemin entre les instances existe, le \proposal pourra être synchronisé permetant au créateur de l'accepter ou de le refuser.}
}

\columnbreak
% arg1 = current distri level
% arg2 = text
\newcommand{\changestyledistribution}[2]{
    \ifthenelse{#1 > #2}{
        \tikzset{currentstyle/.style=fullnode}
    }{
        \tikzset{currentstyle/.style=emptynode}
    }
}
% arg1 = current distri level
\newcommand{\distrigraph}[1]{
    \def \scale {0.2}
    \def \scaletext {0.1}
    \begin{tikzpicture}[
        emptynode/.style={circle, draw=black, scale=\scale},
        fullnode/.style={circle, draw=black, fill=black, scale=\scale},
        textnode/.style={scale=0.7, inner sep=3pt,xshift=-2pt},
    ]
    \tikzset{
        currentstyle/.style={}
    }
        \changestyledistribution{#1}{0}
        \node[currentstyle] (d0)  {};
        \changestyledistribution{#1}{1}
        \node[currentstyle] (d1a) [above right= 1pt and 30pt of d0] {};
        \node[currentstyle] (d1b) [below right= 1pt and 30pt of d0] {};
        \changestyledistribution{#1}{2}
        \node[currentstyle]  (d2)  [right=of d1b] {};
        \changestyledistribution{#1}{2}
        \node[currentstyle]  (d3)  [right=of d2] {};

        \node[textnode] (d0-notice) [above= 10pt of d0] {$n=0$};
        \node[textnode] (d1a-notice) [above= 5pt of d1a] {$n=1$};
        \node[textnode] (d2-notice) [above= 15pt of d2] {$n=2$};
        \node[textnode] (d3-notice) [above= 15pt of d3] {$n=3$};

        \draw[-] (d0) to[out=30, in=180] (d1a);
        \draw[-] (d0) to[out=-30, in=180] (d1b);
        \draw[-] (d1b) -- (d2);
        \draw[-] (d2) -- (d3);

        % \draw[-] (d0-notice.east) -- +(15pt,0pt) -- (d0.135);
        % \draw[-] ($(d0-notice.east) + (-1pt,-2pt)$) -- ($(d0) + (-3pt,2pt)$);
    \end{tikzpicture}
}

\newcommand{\createdistrilegend}{
    \def \scale {0.2}
    \begin{tikzpicture}[
        emptynode/.style={circle, draw=black, scale=\scale},
        fullnode/.style={circle, draw=black, fill=black, scale=\scale},
        textnode/.style={scale=0.7, inner sep=3pt,xshift=-2pt},
    ]
        \node[emptynode] (empty) {};
        \node[fullnode] (full) [below=5pt of empty] {};
        \node[textnode] () [right=3pt of empty] {Does not have the Event};
        \node[textnode] () [right=3pt of full] {Has the Event};
    \end{tikzpicture}
}
\cheatboxlarge[Contrôle qui peut voir les données et comment elles doivent être synchronisées.]{Distribution}{
    \boxentry{Organisation only}{Seulement les membres de l'organisation}
    \boxentry{This community}{Les organisations sur l'instance MISP}
    \boxentry{Connected Communities}{Les organisations sur l'instance et celles d'autres instances se synchronisant dessus. Lorsque les données sont reçues, leur distribution est réduite à \texttt{This community} afin d'éviter d'autres propagations. ($n\leq 1$)}
    \vspace*{-0.7em}
    \begin{center}
        \createdistrilegend
        \hspace*{1em}
        \distrigraph{2}
    \end{center}
    \boxentry{All Communities}{Tous ceux ayant accès. Les données seront propagées librement dans le réseau d'instances connectées. ($n = \infty$)}
    \vspace*{-0.7em}
    \begin{center}\distrigraph{3}\end{center}
    \boxentry{\linkdest{sharinggroup}Sharing Groups}{Liste de distribution qui énumère la liste des organisations ayant accès aux données et comment celles-ci doivent être synchronisées.}

    \begin{multicols*}{2}
        \begin{center}
            \begin{tabular}{| l | l |}
                \hline
                \multicolumn{2}{|c|}{\sharinggroup configuration} \\
                \hline
                \multirow{3}{*}{Organisations} & Org. $\alpha$\\
                    & Org. $\omega$\\
                    & Org. $\gamma$\\
                \hline
                \multirow{3}{*}{Instances*} & MISP 1\\
                    & MISP 2\\
                    & MISP 3\\
                \hline
            \end{tabular}\\
            *Ou activé le mode roaming à la place
        \end{center}
        \columnbreak

        \begin{center}
            \begin{tikzpicture}[
    node/.style={inner sep=0pt},
    simplebox/.style n args={3}{
        rectangle, rounded corners, thick,
        draw=black, fill=#1,
        minimum width=#2,
        minimum height=#3,
        inner sep=10pt, inner ysep=10pt
    },
    simpleboxtitle/.style = {
        rectangle, rounded corners=0,
        minimum width=1em,
        fill=brown!10, text=black, draw, thick,
        font=\bfseries,
        inner sep=3pt
    },
    header/.style = {%
        inner ysep = +1.0em,
        append after command = {
            \pgfextra{\let\TikZlastnode\tikzlastnode}
            node[simpleboxtitle] (header-\TikZlastnode) at (\TikZlastnode.north) {#1}
        }
    },
    coloredHeader/.style n args={2}{
        inner ysep = +1.0em,
        append after command = {
            \pgfextra{\let\TikZlastnode\tikzlastnode}
            node[simpleboxtitle,fill=#2] (header-\TikZlastnode) at (\TikZlastnode.north) {#1}
        }
    }
]

    \node[simpleboxtitle] (orgs01) [] {Org. $\pmb{\alpha}$};
    \node[simpleboxtitle] (orgs02) [below = 0.3em of orgs01] {Org. $\pmb{\omega}$};
    \node[fit = (orgs01) (orgs02)] (orgs0) {};
    \node[simpleboxtitle] (orgs1) [above right= -0.8em and 4em of orgs0] {Org. $\pmb{\omega}$};
    \node[simpleboxtitle] (orgs2) [below = 3.5em of orgs1] {Org. $\pmb{\gamma}$};
    \begin{scope}[on background layer]
        \node[yshift=2pt, fit = (orgs01) (orgs02), simplebox={gray}{1em}{2em}, coloredHeader={MISP 1}{blue!10}] (m1) {};
        \node[yshift=2pt, fit = (orgs1), simplebox={gray}{1em}{2em}, coloredHeader={MISP 2}{blue!10}] (m2) {};
        \node[yshift=2pt, fit = (orgs2), simplebox={gray}{1em}{2em},  coloredHeader={MISP 3}{blue!10}] (m3) {};
    \end{scope}

    \draw[-, thick] (m1) to[out=20, in=180] (m2);
    \draw[-, thick] (m1) to[out=-20, in=180] (m3);
\end{tikzpicture}
        \end{center}
    \end{multicols*}
}

\cheatboxlarge[L'acte de partager où tout le monde peut être un consommateur et/ou un producteur.]{Synchronisation}{
    Une synchronisation dans un sens entre deux instances MISP. L'organisation $\alpha$ avait créé un \textit{sync user} \faicon{user-plus} sur MISP 2 et a noté la clé API générée. Un lien de synchronisation peut être créé sur MISP 1 en utilisant cette clé API et l'organisation du \textit{sync user}. Dès lors, MISP 1 peut \textit{pull} des données depuis MISP 2 et \textit{push} des données vers MISP 2.
    \begin{center}
        \pgfdeclarelayer{bg0}    % declare background layer
\pgfdeclarelayer{bg1}    % declare background layer
\pgfsetlayers{bg0,bg1,main}  % set the order of the layers (main is the standard layer)
\begin{tikzpicture}[
    simplebox/.style n args={3}{
        rectangle, rounded corners, thick,
        draw=black, fill=#1,
        minimum width=#2,
        minimum height=#3,
        inner sep=10pt, inner ysep=10pt
    },
    simpleboxtitle/.style = {
        rectangle, rounded corners=0,
        minimum width=1em,
        fill=brown!10, text=black, draw, thick,
        font=\bfseries,
        inner sep=3pt
    },
    header/.style = {%
        inner ysep = +1.0em,
        append after command = {
            \pgfextra{\let\TikZlastnode\tikzlastnode}
            node[simpleboxtitle] (header-\TikZlastnode) at (\TikZlastnode.north) {#1}
        }
    },
    coloredHeader/.style n args={2}{
        inner ysep = +1.0em,
        append after command = {
            \pgfextra{\let\TikZlastnode\tikzlastnode}
            node[simpleboxtitle,fill=#2] (header-\TikZlastnode) at (\TikZlastnode.north) {#1}
        }
    },
    user/.style = {
        inner sep=0pt
    },
    legend/.style = {
        rectangle, rounded corners=0,
        inner sep=2pt,
        draw=black
    },
    nodes = {align = center}
]

    \node[user] (misp1users) {\faicon{user} \faicon{user} \faicon{user}};
    \node[user] (misp2users) [right= 13em of misp1users] {\faicon{user} \faicon{user}};
    \node[user] (misp2users2) [right= 3em of misp2users] {\faicon{user} \faicon{user} \faicon{user}};
    \node[user,inner xsep=3pt] (syncuser) [left= 0.0em of misp2users] {\faicon{user-plus}};
    \begin{pgfonlayer}{bg1}
        \node[yshift=2pt, fit = (misp1users), simplebox={white}{1em}{2em}, header = Org. $\pmb{\alpha}$] (misp1org) {};
        \node[yshift=2pt, fit = (misp2users) (syncuser), simplebox={white}{1em}{2em}, header = Org. $\pmb{\alpha}$] (misp2org) {};
        \node[yshift=2pt, fit = (misp2users2), simplebox={gray!30}{1em}{2em}, header = Org. $\pmb{\omega}$] (misp2org2) {};
    \end{pgfonlayer}
    \begin{pgfonlayer}{bg0}
        \node[yshift=+0.5em, fit = (misp1org), simplebox={gray}{11em}{6em}, coloredHeader={MISP 1}{blue!10}] (m1) {};
        \node[yshift=+0.5em, fit = (misp2org) (misp2org2), simplebox={gray}{11em}{6em}, coloredHeader={MISP 2}{blue!10}] (m2) {};
    \end{pgfonlayer}
    \draw[-,very thick] (m1.south) -- ++(0,-15pt) -| ($(syncuser.south) + (0,-2pt)$) node [
        pos=0.25,above,yshift=-0.9em
    ] (textsync) {Sync. connection};
    \def \offsetY {3}
    \draw[->,thick] ($(m1.east) + (1pt,\offsetY pt)$) -- ($(m2.west) + (-1pt,\offsetY pt)$) node [above,midway,yshift=-\offsetY pt] {PUSH};
    \draw[<-,thick] ($(m1.east) + (1pt,-\offsetY pt)$) -- ($(m2.west) + (-1pt,-\offsetY pt)$) node [below,midway,yshift=\offsetY pt] {PULL};
\end{tikzpicture}
    \end{center}
}
\end{multicols*}
